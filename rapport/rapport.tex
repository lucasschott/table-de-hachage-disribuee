\documentclass[a4paper,11pt,DIV=12]{scrreprt}
\usepackage[utf8]{inputenc}
\usepackage[T1]{fontenc}
\usepackage[francais]{babel}
\usepackage{graphicx}
\usepackage[export]{adjustbox}
\usepackage{setspace}
\usepackage{dirtytalk}
\usepackage{amsmath}
\usepackage[lined,boxed,commentsnumbered, ruled,vlined,linesnumbered, french]{algorithm2e}
\usepackage{algorithmic}
\usepackage{bytefield}
\usepackage{xcolor}
\usepackage{listings}
\definecolor{codegray}{rgb}{0.92,0.92,0.92}

\lstset{basicstyle=\ttfamily\small,
backgroundcolor=\color{codegray},   
showstringspaces=false,
commentstyle=\color{red},
keywordstyle=\color{blue}
}

\renewcommand{\algorithmicrequire}{\textbf{Require:}}
\renewcommand{\algorithmicensure}{\textbf{Ensure:}}
\renewcommand{\algorithmiccomment}[1]{\{#1\}}
\renewcommand{\algorithmicend}{\textbf{fin}}
\renewcommand{\algorithmicif}{\textbf{si}}
\renewcommand{\algorithmicthen}{\textbf{alors}}
\renewcommand{\algorithmicelse}{\textbf{sinon}}
\renewcommand{\algorithmicelsif}{\algorithmicelse\ \algorithmicif}
\renewcommand{\algorithmicendif}{\algorithmicend\ \algorithmicif}
\renewcommand{\algorithmicfor}{\textbf{pour}}
\renewcommand{\algorithmicforall}{\textbf{pour tout}}
\renewcommand{\algorithmicdo}{\textbf{faire}}
\renewcommand{\algorithmicendfor}{\algorithmicend\ \algorithmicfor}
\renewcommand{\algorithmicwhile}{\textbf{tant que}}

\title{Rapport de projet: table de hachage distribuée}
\author{Lucas SCHOTT et Juan Manuel TORRES GARCIA}


\begin{document}

%\begin{onehalfspace}

    \maketitle

    \chapter{Introduction}

    \section{Présentation}

    Le but de ce projet était de réaliser une bibliothèque de gestion de table
    de hachage distibuée. La table permet de stocker des hashs d'une taille
    de 72octets et d'y associer une ou plusieures adresses IPv6. En faisant
    l'analogie avec le systeme de fichiers torrents, c'est comme si la table
    stockait les  adresses IPv6 des hôtes qui possèdent le fichier
    correspondent au hash donnée.

    Nos servers stockent des adresses associées à des hashs, et les clients
    contactent les servers pour leur transmettre des associations
    hash/adresse, ou pour récuperer des adresses à partirs de hash.
    \\

    \section{fonctionnement}

    Un client peut contacter un server et lui envoyer un couple
    hash/adresse\_IPv6, ce server stock ces données dans sa table. Un hash
    peut être associé à plusieurs adresses. Un client peut aussi contacter un
    server pour lui demander les adresses associés à un hash donné, le server
    lui répond en lui indiquant les élément correspondant trouvé dans sa base
    de données.

    Les servers maintiennent leur base de donnée propre, ils vérifient
    périodiquement qu'il n'y ai pas de données obsolètes et les suppriment
    s'ils en trouvent. Un hash est considéré comme obsolète si aucun server
    n'a reçu de message d'insertion de donnée concernant ce hash depuis plus
    de 30 secondes.
    \\

    Plusieurs servers peuvent se connecter entre eux, au moment de la
    connection de deux servers ceux-ci s'échangent toutes les données qu'ils
    possèdent. Ensuite les servers s'envoient mutuellement les listes des
    servers qu'il connaissent, et les servers vont ainsi démarrer de nouvelles
    procédure de connexion avec les autres servers. Tous les servers sont
    connectés entre eux avec une topologie full-mesh. Ainsi quand un server
    reçoit une donnée d'un client, il va transmettre ces données à tous les
    autres servers. 

    Ainsi à tout moment, si un client envoie des données à un server qui est
    connecté à un groupe de server, ces données se retrouvent dans tous les
    servers. Et quelque soit le server contacté, le client recevra les même
    informations.

    Les servers connectés s'envoient des messages de manière périodique pour
    maintenir leur connection, si un server ne reçoit plus de messages d'un
    autre server il va le conséderer comme mort et les deux servers se
    déconnecteront. Un server est considéré comme mort au bout de 30 secondes
    sans réponse.

    Si un server reçoit une demande d'adresses d'un hash qu'il n'a pas
    dans sa base de donnée, avant de répondre négativement à la requête du
    client, le server demande aux autres servers s'ils ne disposent pas de
    tels informations, puis il répond au client en conséquence.

    \chapter{Utilisation}

    \ \\ \\

    \begin{lstlisting}
    $ ./server <adresse_IPv6|nom_de_domaine> <numero_de_port> [-v]
    \end{lstlisting}

    Sert à creer un server unique, en lui donnant une adresse IPv6 à utiliser
    disponible sur la machine courante et un numéro de port à utiliser,
    ce server peut stocker des hashs associés à des adresses et recevoir et
    envoyer des données à des clients.
    \\ \\

    \begin{lstlisting}
    $ ./server <adresse|nom_de_domaine> <numero_de_port>
    <adresse|nom_de_domaine> <numero_de_port> [-v]
    \end{lstlisting}

    Permet de creer un server comme précédement et de le connecter à un autre
    server déjà existant, en lui donnant en plus de sa propre adresse et
    numéro de son port, l'adresse (ou le nom de domaine) et
    le numéro de port du server auquel se connecter. Le nouveau server
    communiquera avec celui déjà en place et intègrera le groupe de servers
    dans lequel le server contacté est déjà intégré.
    \\ \\

    \begin{lstlisting}
    $ ./client  <adresse_IPv6|nom_de_domaine> <port>
    get <hash> [-v]
    \end{lstlisting}

    Permet de contacter un server en lui demandant les adresses Iv6 qui sont
    associés au hash demandé.
    \\ \\

    \begin{lstlisting}
    $ ./client  <adresse_IPv6|nom_de_domaine> <port>
    put <hash> <adresse_IPv6> [-v]
    \end{lstlisting}

    Permet d'envoyer une adresse associé à un hash à un server.
    \\ \\

    Les deux programmes, \emph{client} comme \emph{server} peuvent prendre
    l'option \emph{-v} (verbose) en argument. Avec cette option activée le
    programme affiche explicitement ses différentes actions, et plus
    particulièrement les détails de synchonisations entre server, et les
    émissions et réceptions de messages.

    \chapter{Structures de données}

    \section{Table de hachage}

    Dans le server la table contenant les hashs et les adresses est un AVL où
    chaque noeud correspond à un hash. Chaque noeud est lié à une liste
    chainée contenant les adresses IPv6 associés à ce hash.\\
    
    \includegraphics[scale=0.15,center]{hash_table_schema.png}

    L'avantage d'un AVL par rapport à un taleau est que l'insertion et la
    suppression des éléments est très simple et très rapide. Or cette
    application passe son temps à inserer des données, et à supprimer les
    données obsolètes. La recherche d'un élément dans un AVL est en log(n) ce
    qui est très convenable.

    \section{Format de messages}


    Il y a trois formats de messages:
    \begin{itemize}
        \item Le format \emph{get\_t} contenant un en-tête et un hash. Celui-ci
            est utilisé quand le client ou un server demandes des adresses à
            un autre server.\\

            \begin{bytefield}{40}
                \bitheader[lsb=0]{0,7,8,15,16,23,24,31,32,39} \\
                \bitbox{16}{2 octets: en-tête} & \bitbox[lrt]{24}{} \\
                \wordbox[lr]{1}{72 octets: hash} \\
                \skippedwords \\
                \wordbox[lrb]{1}{} \\
            \end{bytefield}
        \item Le format \emph{put\_t} contenant un en-tête, un hash et une
            adresse. Celui-ci est utilisé quand des servers s'échangent des
            données, quand un client envoie des données à un server ou quand un
            server répond à un client.\\

            \begin{bytefield}{40}
                \bitheader[lsb=0]{0,7,8,15,16,23,24,31,32,39} \\
                \bitbox{16}{2 octets: en-tête} & \bitbox[lrt]{24}{} \\
                \bitbox[lr]{40}{16 octets: adresse IPv6} \\
                \bitbox[lr]{40}{} \\
                \bitbox[lrb]{24}{} & \bitbox[lrt]{16}{} \\
                \wordbox[lr]{1}{72 octets: hash} \\
                \skippedwords \\
                \wordbox[lrb]{1}{} \\
            \end{bytefield}

        \item Et le format \emph{syn\_t} contenant un en-tête, une adresse et
            un numero de port. Celui-ci est utilisé quand un server envoie des
            messages de synchronisation à un autre server.\\

            \begin{bytefield}{40}
                \bitheader[lsb=0]{0,7,8,15,16,23,24,31,32,39} \\
                \bitbox{16}{2 octets: en-tête} & \bitbox[lrt]{24}{} \\
                \bitbox[lr]{40}{16 octets: adresse IPv6} \\
                \bitbox[l]{24}{} & \bitbox[rb]{16}{} \\
                \bitbox[lrb]{24}{} & \bitbox{16}{2 octets: no de port} \\
            \end{bytefield}

    \end{itemize}


    \chapter{Implémentation}


    \section{Échanges client/server}


    \subsection{Client}

    \begin{algorithm}
        \begin{algorithmic}[H]
            \STATE {ouverture de la socket vers le server}
            \IF{le message est put}
                \STATE {envoyer les données au server}
            \ELSIF{le message est get}
                \STATE {envoyer la requête au server}
                \STATE {attendre la réponse du server pendant une seconde}
                \STATE {afficher la réponse}
            \ENDIF
        \end{algorithmic}
        \caption{Client: algorithme principal}
        \label{alg:client} 
    \end{algorithm}


    \subsection{Server}
    
    \begin{algorithm}
        \begin{algorithmic}[H]
            \WHILE{le server est ouvert}
                \STATE{reception de message}
                \IF{le message est put}
                    \STATE {stocker les données dans la table}
                \ELSIF{le message est get}
                    \STATE {chercher dans la table}
                    \IF{le hash est dans la table}
                        \STATE {envoyer les données au client}
                    \ELSE 
                        \STATE {demander aux autres servers}
                        \IF {les autres servers envoyent des données}
                            \STATE {envoyer ces données au client}
                        \ELSE
                            \STATE {ne rien n'envoyer}
                        \ENDIF
                    \ENDIF
                \ENDIF
            \ENDWHILE
        \end{algorithmic}
        \caption{Server: réponse au client}
        \label{alg:server} 
    \end{algorithm}


    \section{Maintient de la table}

    \begin{algorithm}
        \begin{algorithmic}[H]
            \WHILE{le server est ouvert}
                \STATE{reception de message}
                \IF{le message est put data}
                    \STATE {stocker les données dans la table et mettre à jour
                    la date de dernier put}
                \ELSIF{le message est share data}
                    \STATE {stocker les données dans la table sans mettre à jour
                    la date de dernier put}
                \ELSIF{le message est get data}
                    \STATE {chercher dans la table}
                    \IF{le hash est dans la table}
                        \STATE {envoyer les données au server}
                    \ELSE 
                        \STATE {ne rien n'envoyer}
                    \ENDIF
                \ENDIF
            \ENDWHILE
        \end{algorithmic}
        \caption{Server thread1: réceptions des données d'autres servers}
        \label{alg:server} 
    \end{algorithm}

    \ \\ \\ \\
    
    \begin{algorithm}
        \begin{algorithmic}[H]
            \WHILE{le server est ouvert}
                \WHILE{parcours de chaque hash de la table}
                    \IF{la date de dernier put est ancienne de plus de 30
                    secondes}
                        \STATE{supprimer le hash de la table}
                    \ENDIF
                \ENDWHILE
                \STATE attendre 5 secondes
            \ENDWHILE
        \end{algorithmic}
        \caption{Server thread2: maintient de la table propre}
        \label{alg:server} 
    \end{algorithm}

    \newpage


    \section{Connexion des servers}

    \begin{algorithm}
        \begin{algorithmic}[H]
            \WHILE{le server est ouvert}
                \STATE{reception de message}
                \IF{le message est server connect}
                    \STATE{lui envoyer tous les servers auquels on est déjà
                    connecté}
                    \STATE{l'enregistrer dans la table des servers connectés et
                    initialiser la date de dernier keep\_alive}
                    \STATE{lui envoyer un message connect}
                    \STATE{lui envoyer toutes les données que l'on possède dans
                    la table de hachage}
                \ELSIF{le message est keep}
                    \STATE{mettre a jour la date de dernier keep\_alive de ce server}
                \ELSIF{le message est share server}
                    \STATE{ajouter le server dans la table des servers connus
                    et initialiser la date de dernier keep\_alive}
                    \STATE{lui envoyer un message connect}
                \ENDIF
            \ENDWHILE
        \end{algorithmic}
        \caption{Server thread1: connexion et partage de données entre servers}
        \label{alg:server} 
    \end{algorithm}
    
    \begin{algorithm}
        \begin{algorithmic}[H]
            \WHILE{le server est ouvert}
                \STATE{envoyer un message keep\_alive à tous les servers connectés}
                \WHILE{parcours de chaque server de la table des servers
                connus}
                    \IF{la date de dernier keep\_alive est ancienne de plus de
                    30 secondes}
                        \STATE{supprimer le server de la table}
                    \ENDIF
                \ENDWHILE
                \STATE attendre 5 secondes
            \ENDWHILE
        \end{algorithmic}
        \caption{Server thread2: maintient des connections avec les servers}
        \label{alg:server} 
    \end{algorithm}

    \chapter{Conclusion}

    Ce projet à été très intéressant à réaliser, c'est surement le projet qui
    nous à le plus passionnée pendant ce semestre. Il a demandé un très grand
    investissement car nous y avons passé de très nombreuses heures à réflichire
    aux structures de données, à la communication entre les servers, à
    l'utilisation des sockets en IPv6 et à l'implémentation de tout cela.

    Nous avons aussi réaliser un scripte de test qui teste toutes les
    situations les plus courantes d'utilisation des servers et des clients,
    avec de nombreux servers connectées entre eux et de multiples messages
    echangée. Au final toutes les fonctionnnalités décrites dans le sujet 
    fonctionnent et nous assez satisfait de notre travail.

    Il reste néanmoins des points que nous aurions pu améliorer mais
    cela n'a pas été possible par manque de temps. Par exemple nos programme
    ne sont pas compatible IPv4. Et les servers étant connectés avec une
    topologie full-mesh, énormament de messages sont échangés entre les
    servers, que ce soit rien que  pour pour le maintient des connections entre
    servers, mais aussi quand un server cherche une donnée, il va demander à
    tous les autres servers, et s'ils l'ont ils vont tous lui répondre le même
    message.

%\end{onehalfspace}

\end{document}
